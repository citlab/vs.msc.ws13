\section{Introduction}
\label{sect:introduction}

The amount of data stored world wide grows continuously -- \BigData\ is everywhere. The term \BigData\ refers to the practice of using large data sets from multifarious sources and processing them with high speed and efficiency to create valuable business information. Classical relational databases are often not capable of processing such large volumes of data. Thus, a new type of software is exploited by \BigData, that runs on hundreds or thousands of machines in parallel, like \Stratosphere\ by the \textsl{\ac{TUB}} or \textsl{Hadoop}, the open-source implementation of Map-Reduce. Until now, many recent platforms are restricted to process static data sets stored on distributed file systems. These system do not meet challenges of real-time processing tasks, that handle data streams generated by the world-wide-web or sensor networks.

The group \textsl{\ac{CIT}} at the \textsl{\ac{TUB}} aims to fill that gap by exploring how \BigData-platforms can be extended by the capability of processing data streams. As a first step in this direction, this work contributes by basic research on how well known operators like map, reduce or join, are applicable to the context of stream-processing. Section \ref{sect:operators} lists and defines the operators that are scoped by this work. blabla structure

%Die weltweiten Datenbestände wachsen stetig – Big Data ist in aller Munde. Big Data bezeichnet den Einsatz großer Datenmengen aus vielfältigen Quellen mit einer hohen Verarbeitungsgeschwindigkeit zur Erzeugung wirtschaftlichen Nutzens. Klassische relationale Datenbanksysteme sind oft nicht in der Lage, derart große Datenmengen zu verarbeiten. Deswegen kommt für Big Data eine neue Art von Software zum Einsatz, die parallel auf bis zu Hunderten oder Tausenden von Prozessoren bzw. Servern arbeitet, z.B. das Stratosphere System der TU Berlin oder das Quelloffene Hadoop. Aktuelle Plattformen sind bisher darauf beschränkt nur statische Daten (gespeichert in einem verteilten Dateisystem) zu verarbeiten. Die Systeme sind nicht in Laage mit der neuen Herausforderung der Echtzeitdatenverarbeitung in Form von Daten-Streams umzugehen, die z.B. aus dem Internet stammen oder durch Sensornetzwerke generiert werden.
%
%Das CIT will diese Lücke schließen und untersuchen wie Big-Data Plattform für Streamprocessing erweitert werden können. Dafür müssen Grundlagen erforscht werden, wie bekannte Operatoren wie Map, Reduce, Join,... für die Stream-Datenverarbeitung erweitert werden können.
%
%Im Rahmen des Projektes sollen verschiedenste Operatoren untersucht und für das Streamprocessing erweitert werden. Diese Stream-Operatoren sollen dann auf Basis einer offenen Plattform (vorauss. Twitter Storm) prototypisch implementiert, evaluiert, und auf deren inhärente Eigenschaften untersucht werden.
 
	\hfill \today