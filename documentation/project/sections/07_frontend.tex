\section{Frontend}
\label{sect:frontend}

To have an easy way to start and test our topologies, we decided to develop a frontend. So we defined some requirements. The first was, that it should be accessable as easy as possible. So we wrote the Cit-Storm webinterface.

\subsection{Requirements}
We defined some requirements for our cluster, the deployment section and security.

For Securtiy, we developed an esay user mangament for login.

Every User should be able to start and stopp the Nimbus and Cassandraserver. For Supervisor starup an amount of servers can ba set.

A topology can be uploaded over the interface and is linked to the uploading user. Topologies can started with a single click after uploading.

To see if the topology runs the user can start the storm ui or filter for logs.

\subsection{Builidng up Cit-Storm Webinterface}
Because of the complexity of a modern webinterface we used some libraries. As the basic, we used the play Framework. So we can easily start a webserver and develop our requirements in an iterativ and agile process.

For easy and browser compltibility we used jquery with some plugins for ajax request with forms. Styling is set by default bootstrap with some adjustments and own styles.

But first we need a simple user management to be sure that not everyone cann start our Cluster or upload bad jar files and start theme.

Users are stored in a mysql database with user name and hashed passwort. A user need to be logged in to work with the webinterface.

\subsection{Using Cit Strom Webinterface}
After login you will see a dashboard with some infos about the cluster and uploaded topologies.

If a topologie should run, a nimbus, cassandra and at least one supervisor should run. Over the nimbus is the strom ui accessable. There you get an overview if a topology is allready running.

Starting a part of the cluster is realised over a small webservice which uses a process builder to start the server with aws tools. The status will be update every five seconds an uses the aws tools and parse the json respond.

Topology are stored directly on the nimbus. After click start topology, it will be started and distributed to the supervisors.

Generated logs from the running topologies are stored in the mysql database. Because of the big amount of logs, we need to filter the logs. After typing a text in the filterinput and press enter the logs will update every second.
