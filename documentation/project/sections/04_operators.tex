\section{Operators}
\label{sect:operators}
This sections lists and describes the operators that are discussed in this work.
For each operator a common description is given about its traditional behavior plus concerns regarding how its definition changes when applied into a stream-processing context.
\begin{description}
  \item[map] \hfill \\
  The \map\ operator is most often defined as a function that converts a data set by applying a higher-order function \textsl{\_map} against each element of the set, resulting in a new data set of modified elements \cite{Wiki:Map}:
	$$map\left(S, \_map\right)=\{s\in S : \_map(s)\}$$
	This concept stays basically the same when \map\ is applied in a stream-processing context except that its inputs and outputs are streams of elements not sets.
  \item[filter] \hfill \\
  The \filter\ operator is commonly understood as a function that converts a data set to a subset by removing all elements from the original data set that do not match a higher-order predicate \textsl{\_filter}\cite{Wiki:Filter}:
	$$filter\left(S, \_filter\right)=\{s\in S\ | \_filter\left(s\right)\}$$
	This concept stays basically the same when \filter\ is applied in a stream-processing context except that its inputs and outputs are streams of elements not sets.
  \item[reduce] \hfill \\
  The third etc \ldots
	\item[join] \hfill \\
  The third etc \ldots
	\item[groupby] \hfill \\
  The third etc \ldots
	\item[update] \hfill \\
  The third etc \ldots
\end{description}

%\subsection{Operator Types}
%\label{sect:operatorTypes}